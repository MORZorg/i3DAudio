\documentclass{beamer}

\usepackage[utf8]{inputenc}
\usepackage{graphicx}
\usepackage{units}
\usepackage{mathtools}

%\usetheme{Copenhagen}
\usetheme{Antibes}

% Footer with page number
\setbeamertemplate{footline}%{miniframes theme}
{
	\begin{beamercolorbox}[colsep=1.5pt]{upper separation line foot}
	\end{beamercolorbox}

	\begin{beamercolorbox}[ht=2.5ex,dp=1.125ex,
		leftskip=.3cm,rightskip=.3cm plus1fil]{author in head/foot}
		\leavevmode{\usebeamerfont{author in head/foot}\insertshortauthor}
		\hfill
		\insertframenumber
	\end{beamercolorbox}

	\begin{beamercolorbox}[colsep=1.5pt]{lower separation line foot}
	\end{beamercolorbox}
}

% Other stuff
\hypersetup{pdfstartview={Fit}}
\setbeamertemplate{caption}[numbered]

\title{3D audio source simulation on iOS devices}
\author[Orizio, Rizzini, Zucchelli]{Orizio Riccardo,\\Rizzini Mattia,\\Zucchelli Maurizio}
\date{xx july 2014}
\institute[UniBS]{University of Brescia}
\logo{\includegraphics[width=15mm]{images/logo.png}}

\begin{document}
	\begin{frame}
		\maketitle
	\end{frame}

	\section{Overview}
	
	\begin{frame}
		\frametitle{\insertsection}
		The project goal is to realize a 3D audio simulator for the iPad, in
		which an audio source can be moved using a graphical interface [or it follows
		the user's head orientation automatically].\\
	\end{frame}

	\begin{frame}
		\frametitle{\insertsection}
		The app is written in multiple languages:
		\begin{itemize}
			\item {\bf Ruby}, {\bf Matlab} A {\em Ruby} and a {\em Matlab} scripts are used to 
			preprocess the Database
			\item {\bf Pure Data} The audio I/O is managed using a {\em Pure Data} patch
			\item {\bf C++} The functional core of the patch is a PD external written entirely
				by us
			\item {\bf Swift} The new Apple's language for iOS devices is used to develop the
				graphical inferface of the app, which communicates with the PD patch
		\end{itemize}
	\end{frame}

	\AtBeginSection[]
	{
		\begin{frame}
			\frametitle{Outline}
			\tableofcontents[currentsection]
		\end{frame}
	}

	\section{Database Preprocessing}

	\begin{frame}
		\frametitle{\insertsection}
		The KEMAR database is a list of HRTF recorded using a manikin with two
		microphones in place of the ears.\\
		Each HRTF is a couple of 128-samples FIR filters (one per ear), associated to the position
		of the source at the moment of recording.\\
	\end{frame}

	\begin{frame}
		\frametitle{\insertsection}
		The database is a textual file containing the data.\\
		The database is processed offline, using two scripts that create a C++ header in which the data
		are defined as instance of two global vectors and an additional vector containing the result
		of the {\em Delaunay triangulation}.\\
	\end{frame}

	\subsection{Delaunay Triangulation}

	\begin{frame}
		\frametitle{\insertsection - \insertsubsection}
		It is an algorithm that, given a set of points in a space, determines a subdivision in triangles
		having the points as vertices, such that no point is left inside a triangle.\\
		This subdivision allows us to search for the three points that are closes to the source at any time
		with less effort, compared to a brute force search.\\
	\end{frame}

	\begin{frame}
		\frametitle{\insertsection - \insertsubsection}
		The subdivision is performed thanks to a Matlab script.\\
		The script is called by the Ruby one, which saves the result in the C++ header, alongside
		the original data.
	\end{frame}

	\begin{frame}
		\frametitle{\insertsection - \insertsubsection}
		// TODO insert Matlab subdivision
	\end{frame}

	\section{Pure Data patch}

	\begin{frame}
		\frametitle{\insertsection}
		//TODO insert patch screen here
	\end{frame}

	\begin{frame}
		\frametitle{\insertsection}
		The input signal, along with the source's actual azimuth and elevation, are provided
		as inlets to the block orz$\_$hrtf$\sim$, which produces as outlets the
		left and right channel of the filtered signal.\\
		The whole filtering is performed by that block, implemented as C++ external.\\
	\end{frame}

	\section{Processing external}

	\begin{frame}
		\frametitle{\insertsection}
		A Pure Data external is structured in a very precise manner and must be	written in C (or wrapped in C++).\\
		A {\em setup} method is called when the block is loaded in the patch, and it must initialize inlets,
		outlets and callback methods.\\
		A {\em new} method instantiates the internal data of the class after its creation by setup.\\
		A callback method to handle the DSP signal must be present, in our case it is called {\em perform}.\\
	\end{frame}

	\begin{frame}
		\frametitle{\insertsection}
		The perform method receives the signal and azimuth, elevation and range parameters,
		then proceeds in three separate steps:
		\begin{itemize}
			\item Find the points that form the triangle which encloses the source
			\item Determine the coefficients for the HRTF interpolation and calculate the HRTF in the source's
				position
			\item Filter the signal separately with the left and right HRTFs
		\end{itemize}
	\end{frame}

	\subsection{HRTF interpolation}

	\begin{frame}
		\frametitle{\insertsection - \insertsubsection}
		We use the distance between the triangles and the source as a first estimate of the probability of enclosing
		the source, thus reducing the set of triangles in which we have to search the correct one.\\
		The correct one will be the one that produces positive coefficients for the source's HRTF interpolation.\\
	\end{frame}

	\begin{frame}
		\frametitle{\insertsection - \insertsubsection}
		The interpolation is performed as linear combination of the HRTFs of the points with the coefficients, computed as
		\\//CMON equation Y DONT U WORK\\
%		\begin{equation} g = H^{-1} \cdot s \end{equation}
		and normalized to their own summation.\\
		Where H is a matrix having the points' coordinates as column, and s is a vector containing the source's
		coordinates.\\
	\end{frame}
	
	\subsection{Filtering}

	\begin{frame}
		\frametitle{\insertsection - \insertsubsection}
		The filtering is done by performing a convolution of the signal with the two computed HRTFs.\\
		Past filters are kept as external's attributes to smoothen the transition from one filter to another\\
	\end{frame}

	\section{The iPad interface}

	\begin{frame}
		\frametitle{\insertsection}
		\begin{figure}
			\centering
			  \includegraphics[width=0.6\textwidth]{images/marine.jpg}
		\end{figure}
	\end{frame}

	\begin{frame}
		\frametitle{\insertsection}
	\end{frame}

	\begin{frame}
		\frametitle{\insertsection}
	\end{frame}

\end{document}
