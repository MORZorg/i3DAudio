\documentclass{beamer}

\usepackage[utf8]{inputenc}
\usepackage{graphicx}
\usepackage{units}
\usepackage{mathtools}

%\usetheme{Copenhagen}
\usetheme{Antibes}

% Footer with page number
\setbeamertemplate{footline}%{miniframes theme}
{
	\begin{beamercolorbox}[colsep=1.5pt]{upper separation line foot}
	\end{beamercolorbox}

	\begin{beamercolorbox}[ht=2.5ex,dp=1.125ex,
		leftskip=.3cm,rightskip=.3cm plus1fil]{author in head/foot}
		\leavevmode{\usebeamerfont{author in head/foot}\insertshortauthor}
		\hfill
		\insertframenumber
	\end{beamercolorbox}

	\begin{beamercolorbox}[colsep=1.5pt]{lower separation line foot}
	\end{beamercolorbox}
}

% Other stuff
\hypersetup{pdfstartview={Fit}}
\setbeamertemplate{caption}[numbered]

\title{3D audio source simulation on iOS devices}
\author[Orizio, Rizzini, Zucchelli]{Orizio Riccardo, Rizzini Mattia, Zucchelli Maurizio}
\date{1 August, 2014}
\institute[UniBS]{University of Brescia}
\logo{\includegraphics[width=15mm]{images/logo.png}}

\begin{document}
	\begin{frame}
		\maketitle
	\end{frame}

	\section{Overview}
	
	\begin{frame}
		\frametitle{\insertsection}
		The project goal is to realize a 3D audio simulator for the iPad.
		\begin{itemize}
			\item The interface allows to move the audio source around
			\item The interface also allows to change the orientation of the user (manipulating yaw,
				pitch and roll of the head)
		\end{itemize}
	\end{frame}

	\begin{frame}
		\frametitle{\insertsection}
		The app is written in multiple languages:
		\begin{description}[leftmargin=1em]
			\item[Ruby, MATLAB] A {\em Ruby} and a {\em MATLAB} scripts are
				used to pre-process the Database
			\item[Pure Data] The audio I/O is managed using a {\em Pure Data} patch
			\item[C++] The functional core of the patch is a PD external written entirely
				by us
			\item[Swift] The new Apple's language for iOS devices is used to develop the
				graphical interface of the app, which communicates with the PD patch
		\end{description}
	\end{frame}

	\AtBeginSection[]
	{
		\begin{frame}
			\frametitle{Outline}
			\tableofcontents[currentsection]
		\end{frame}
	}

	\section{Database Preprocessing}

	\begin{frame}
		\frametitle{\insertsection}
		\begin{itemize}
			\item The \textsc{KEMAR} database is a list of \textsc{HRTF} recorded using a manikin with two
				microphones in place of the ears
			\item Each \textsc{HRTF} is a couple of 128-samples FIR filters (one per ear), associated to the position
				of the source at the moment of recording
		\end{itemize}
	\end{frame}

	\begin{frame}
		\frametitle{\insertsection}
		\begin{itemize}
			\item The database is a textual file containing the data
			\item It is processed off-line and translated in three vectors containing the
              points' position, their \textsc{HRTF} and the result of the {\em Delaunay Triangulation}
		\end{itemize}
	\end{frame}

	\subsection{Delaunay Triangulation}

	\begin{frame}
		\frametitle{\insertsection\ - \insertsubsection}
		\begin{itemize}
			\item Determines a subdivision of the points' space in triangles
			\item Each triangle has the points as vertices, such that no point is left inside a triangle
			\item This subdivision allows us to search for the three points that determine the triangle enclosing the
				source with little effort
			\item The subdivision is performed thanks to a MATLAB script called by the Ruby one
		\end{itemize}
	\end{frame}

	\begin{frame}
		\frametitle{\insertsection\ - \insertsubsection}
		\begin{figure}
			\centering
			  \includegraphics[width=0.6\textwidth]{images/triangulation.png}
			  \caption{Triangulation}
		\end{figure}
	\end{frame}

	\begin{frame}
		\frametitle{\insertsection\ - \insertsubsection}
		\begin{itemize}
			\item The triangulation is in 2D, the coordinates are azimuth and elevation
			\item The distance is assumed to be 1 for every point
		\end{itemize}
	\end{frame}

	\section{Pure Data patch}

	\begin{frame}
		\frametitle{\insertsection}
		\begin{figure}
			\centering
			  \includegraphics[width=1.0\textwidth]{images/Test_patch.png}
			  \caption{The patch used to test our external}
			  \label{fig:test}
		\end{figure}
	\end{frame}
	
	\begin{frame}
		\frametitle{\insertsection}
		\begin{itemize}
          \item The block {\em orz\_hrtf$\sim$} filters the signal in the given source's position
			\item The position is given in azimuth, elevation and distance from the user
			\item The resulting outlets are the left and right channel of the filtered signal
		\end{itemize}
	\end{frame}

	\begin{frame}
		\frametitle{\insertsection}
		\begin{figure}
			\centering
			  \includegraphics[width=1.0\textwidth]{images/iOS_patch.png}
			  \caption{The patch used by the app}
			  \label{fig:ios_pd}
		\end{figure}
	\end{frame}

	\begin{frame}
		\frametitle{\insertsection}
		\begin{itemize}
			\item The parameters are received from outside ({\em receive} blocks)
			\item The azimuth and elevation are modified accordingly to
				the given yaw, pitch and roll of the head
			\item A version with a simulated stereo source of the patch has been made
		\end{itemize}
	\end{frame}

	\begin{frame}
		\frametitle{\insertsection}
		\begin{figure}
			\centering
			\includegraphics[width=1.0\textheight]{images/iOS_patch_conversion.png}
			\caption{The sub-patch used to map yaw, pitch and roll \newline into azimuth and elevation}
		\end{figure}
	\end{frame}

	\begin{frame}
		\frametitle{\insertsection}
		\begin{figure}
			\centering
			  \includegraphics[width=1.0\textwidth]{images/iOS_patch_stereo.png}
			  \caption{The simulated stereo patch used by the app}
		\end{figure}
	\end{frame}

	\section{Processing external}
	\subsection{Structure}

	\begin{frame}
		\frametitle{\insertsection\ - \insertsubsection}
		\begin{itemize}
			\item A Pure Data external must be written in C or in wrapped C++
			\item A {\em setup} method is called when the block is loaded in the patch to initialize inlets,
				outlets and callback methods
			\item A {\em new} method instantiates the internal data of the class
			\item A callback method to handle the DSP signal must be present, in our case it is called {\em perform}
		\end{itemize}
	\end{frame}

	\begin{frame}
		\frametitle{\insertsection\ - \insertsubsection}
		The {\em perform} method works this way
		\begin{itemize}
			\item Find the points that form the triangle which encloses the source
			\item Determine the coefficients for the \textsc{HRTF} interpolation
			\item Filter the signal separately with the left and right \textsc{HRTF}s
		\end{itemize}
	\end{frame}

	\subsection{\textsc{HRTF} interpolation}

	\begin{frame}
		\frametitle{\insertsection\ - \insertsubsection}
		\begin{itemize}
			\item The distance between the centre of each triangle and the source is used as an heuristic to find the triangle enclosing the source
			\item The correct one will be the one that produces positive coefficients for the source's \textsc{HRTF} interpolation
		\end{itemize}
	\end{frame}

	\begin{frame}
		\frametitle{\insertsection\ - \insertsubsection}
		$$ \textsc{HRTF}_{s,l,r} = \sum\limits_{i=0}^2 \dfrac{ g_i } { \left( \sum\limits_{j=0}^2 g_j \right) } \cdot \textsc{HRTF}_{i,l,r} $$
		$$ g = H^{-1} \cdot s $$
		$$ H = [ point_0 | point_1 | point_2 ] $$
	\end{frame}
	
	\subsection{Filtering}

	\begin{frame}
		\frametitle{\insertsection\ - \insertsubsection}
		\begin{itemize}
			\item The filtering is a convolution of the signal with the two computed \textsc{HRTF}s
			\item Past filters are used to smoothen the transition from one filter to another
		\end{itemize}
	\end{frame}

	\subsection{Distance}

	\begin{frame}
		\frametitle{\insertsection\ - \insertsubsection}
        \begin{itemize}
          \item The \textsc{HRTF} is considered to be distance invariant for distances higher than one meter
          \item The effect attenuation caused by the distance is simulated using the {\em Inverse Square Law}
            $$ I(r) = \frac{I_{0}}{r^2} $$
          \item For lower distances the whole interpolation process becomes much harder
      \end{itemize}
	\end{frame}

	\section{The iPad interface}

	\begin{frame}
		\frametitle{\insertsection}
		\begin{itemize}
			\item Three views of the user's head and the source
			\item The user's head can be turned and the source moved with a drag and drop
			\item The sound is automatically adjusted to the given position by communicating the
				new parameters to the PD patch
		\end{itemize}
	\end{frame}

	\begin{frame}
		\frametitle{\insertsection}
		\begin{figure}
			\centering
			\includegraphics[width=0.9\textheight]{images/iOS_screenshot_0.png}
			\caption{The app running on an iPad}
			\label{fig:ios_app}
		\end{figure}
	\end{frame}

	\section*{Conclusions}

	\begin{frame}
		\frametitle{\insertsection}
		\begin{figure}
			\centering
			\includegraphics[width=0.55\textwidth]{images/alas.png}
            \caption{Suggested grade: 30}
		\end{figure}
	\end{frame}

\end{document}
